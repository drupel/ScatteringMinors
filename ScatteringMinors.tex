\documentclass{amsart}
\usepackage{amsmath,amssymb,latexsym,color}
\usepackage[margin=1in]{geometry}

\newtheorem{theorem}{Theorem}
\newtheorem{corollary}[theorem]{Corollary}
\newtheorem{definition}[theorem]{Definition}
\newtheorem{lemma}[theorem]{Lemma}
\newtheorem{remark}[theorem]{Remark}

\numberwithin{theorem}{section}

\newcommand{\bfc}{\mathbf{c}}
\newcommand{\bfg}{\mathbf{g}}

\newcommand{\cP}{\mathcal{P}}
\newcommand{\cQ}{\mathcal{Q}}

\newcommand{\fp}{\mathfrak{p}}

\newcommand{\CC}{\mathbb{C}}
\newcommand{\RR}{\mathbb{R}}
\newcommand{\TT}{\mathbb{T}}
\newcommand{\ZZ}{\mathbb{Z}}

\newcommand{\ol}[1]{{\overline{#1}}}

\newcommand{\Aut}{\operatorname{Aut}}
\newcommand{\diag}{\operatorname{diag}}
\newcommand{\into}{\hookrightarrow}
\newcommand{\obeta}{{\overline{\beta}}}
\newcommand{\oi}{{\overline{\imath}}}
\newcommand{\ot}{{\overline{t}}}

\title{Generalized Minors via Scattering Diagrams}

\author{Dylan Rupel}
\author{Salvatore Stella}
\author{Harold Williams}

\begin{document}
  \maketitle
  \begin{abstract}
    We prove that all cluster monomials in acyclic cluster algebras with (doubled) principal coefficients are realized as restrictions of generalized minors to Coxeter double Bruhat cells.
  \end{abstract}

  Sketch of proof:
  \begin{enumerate}
    \item Nathan tells us that all cluster monomials are computed by path ordered products along a path from the native chamber of the $\bfg$-vector to the all green chamber.
      Our first goal thus is to show that the evaluation of a generalized minor on the group element naturally associated to a path from its native chamber to the all green chamber gives the correct cluster monomial.
      Morally, crossing a wall should correspond to changing the element we evaluate on (i.e.\ ``conjugating'' by appropriate upper and lower unipotent elements).
      One technical difficulty we will have to overcome is the precise rule for assigning signs to the factorization parameter monomials present in non-simple coroot subgroup elements, perhaps the tropical signs of the $\bfc$-vectors will be important here.
      Then the result proceeds by an induction with obvious base case: evaluation of a minor on a generic element of the Cartan yields the monomial of initial cluster variables corresponding to the $\bfg$-vector.
      We then need to show that the wall crossing automorphisms for the Lie scattering diagram and the cluster scattering diagram are intertwined by the factorization parameter map.
      The proof of this is hopefully an easy consequence of the [GHKK] Lie algebra formalism for scattering diagrams.
      This result alone would probably be worthy of a paper.
    \item In order to complete these results to a proof of our conjecture, we need to show that there is a path from the native chamber of the weight to the all red chamber whose path ordered product is the identity on the evaluated minor.
      In type $A_3$, this seems to be obvious by inspection (any other finite type will probably be just as easy to see).
      For affine and wild types, the mutation sequence to the all red chamber will often contain infinitely many walls and we need to establish for these walls the correct analogue of the concavity present in projected 3-D scattering diagrams which makes this obvious for type $A_3$.
  \end{enumerate}

  \section{Root Systems and Weyl Groups}
    Fix an $n\times n$ symmetrizable Cartan matrix $A=(A_{ij})$ with symmetrizing matrix $D=\diag(d_1,\ldots,d_n)$, i.e. $DA=(d_iA_{ij})$ is a symmetric matrix.
    We assume that each $d_i^{-1}$ is a positive integer.

    The root lattice $\cQ$ is the rank $n$ lattice spanned by the simple roots $\alpha_i$, $i\in[1,n]$.
    Define a symmetric bilinear form on $\cQ$ by $(\alpha_i,\alpha_j)=d_iA_{ij}$.
    Write $\cQ^\vee$ for the coroot lattice, this is the sublattice of $\cQ$ spanned by $\alpha_i^\vee:=d_i^{-1}\alpha_i$, $i\in[1,n]$.
    Then $(\alpha_i^\vee,\alpha_j)=A_{ij}$ for $i,j\in[1,n]$.

    The weight lattice $\cP=\bigoplus\ZZ\rho_i$ is the dual lattice of $\cQ^\vee$, where we write $\{\rho_i:i\in[1,n]\}$ for the basis dual to $\{\alpha_i^\vee:i\in[1,n]\}$.
    Since $\cQ^\vee\subset\cQ$, the coweight lattice $\cP^\vee$ dual to $\cQ$ is naturally contained in $\cP$.
    More precisely, writing $\{\rho_i^\vee:i\in[1,n]\}$ for the basis of $\cP^\vee$ dual to $\{\alpha_i:i\in[1,n]\}$, we have $\rho_i^\vee=d_i^{-1}\rho_i$.
    We write $\langle\cdot,\cdot\rangle$ for both the natural pairing between $\cP$ and $\cQ^\vee$ and the natural pairing between $\cP^\vee$ and $\cQ$.

    Using the bilinear form $(\cdot,\cdot)$, we obtain a map $\cQ^\vee\to\cP^\vee$ given by $\beta^\vee\mapsto (\beta^\vee,\cdot)$.
    In particular, under this map $\alpha_j^\vee\mapsto \sum a_{ji}\rho_i^\vee$ and $\alpha_j\mapsto \sum a_{ij}\rho_i$.
    We will often implicitly identify $\cQ^\vee$ (resp.\ $\cQ$) with its image inside $\cP^\vee$ (resp.\ $\cP$), this should not lead to any confusion.

    Define automorphisms $s_i:\cQ\to\cQ$, $i\in[1,n]$, by $s_i(\beta):=\beta-(\alpha_i^\vee,\beta)\alpha_i$.
    The Weyl group $W$ is the subgroup of $\Aut(\cQ)$ generated by the automorphisms $s_i$, $i\in[1,n]$.
    Observe that $W$ naturally acts on $\cQ^\vee$, $\cP$, and $\cP^\vee$ as follows:
    \[s_i(\beta^\vee)=\beta^\vee-(\beta^\vee,\alpha_i)\alpha_i^\vee,\qquad s_i(\lambda)=\lambda-\langle\lambda,\alpha_i^\vee\rangle\alpha_i,\qquad s_i(\lambda^\vee)=\lambda^\vee-\langle\lambda^\vee,\alpha_i\rangle\alpha_i^\vee.\]
    Note that the action of $W$ is compatible with the maps $\cQ^\vee\to\cP^\vee$ and $\cQ\to\cP$.

    The set $\Phi_{re}$ of real roots is the collection of all $w(\alpha_i)$, $i\in[1,n]$, for $w\in W$.
    For any $\beta\in\Phi_{re}$, set $\beta^\vee:=2\beta/(\beta,\beta)$.

    Associated to the choice of Coxeter element $c=s_1\cdots s_n$, we have a skew-symmetric bilinear form $\omega:\cQ\times\cQ\to\ZZ$ given by $\omega(\alpha_i,\alpha_j):=d_iA_{ij}$ for $i>j$.
    Write $B=(b_{ij})$ for the $n\times n$ matrix given by $\omega(\alpha_i^\vee,\alpha_j)=B_{ij}$.
    Define the \emph{Euler form} $E:\cQ^\vee\times\cQ\to\ZZ$ by $E(\beta^\vee,\phi)=\omega(\beta^\vee,\phi)-(\beta^\vee,\phi)$.


  \section{Coxeter Double Bruhat Cells}
    Let $G$ be the (derived?) Kac-Moody group associated to the Cartan matrix $A$.
    Write $B_\pm\subset G$ for the standard upper and lower Borel subgroups and $H=B_+\cap B_-$ for the Cartan subgroup.
    Let $N_\pm$ denote the unipotent radical of $B_\pm$.

    For $i\in[1,n]$, write $\varphi_i:SL_2\to G$ for the inclusion of the $i$-th coroot subgroup.
    For $t_i,t_\oi,h_i\in\CC^*$, $i\in[1,n]$, we write
    \[x_i(t_i)=\varphi_i\left(\left[\begin{array}{cc} 1 & t_i\\ 0 & 1\end{array}\right]\right),\qquad x_\oi(t_\oi)=\varphi_i\left(\left[\begin{array}{cc} 1 & 0\\ t_\oi & 1\end{array}\right]\right), \qquad h_i^{\alpha_i^\vee}=\varphi_i\left(\left[\begin{array}{cc} h_i & 0\\ 0 & h_i^{-1}\end{array}\right]\right).\]
    For any root $\beta\in\Phi$, we also have an associated coroot $SL_2$ subgroup in $G$.
    For $t\in\CC^*$, write $x_\beta(t)$, $x_{\ol{\beta}}(t)$, $t^{\beta^\vee}$ for the associated one-parameter upper unipotent, lower unipotent, and Cartan elements in $G$.

    Given a Coxeter element $c=s_1\cdots s_n$ of $W$, write $G^{c,c^{-1}}=B_+ c B_+\cap B_- c^{-1} B_-$ for the \emph{Coxeter double Bruhat cell}.
    Elements of the form 
    \begin{equation}
      \label{eq:generic element}
      x_{\ol{1}}(t_{\ol{1}})\cdots x_{\ol{n}}(t_{\ol{n}}) h_1^{\alpha_1^\vee}\cdots h_n^{\alpha_n^\vee} x_n(t_n)\cdots x_1(t_1)
    \end{equation}
    with $t_i,t_\oi,h_i\in\CC^*$, $i\in[1,n]$, form a dense open subset of $G^{c,c^{-1}}$.
    In particular, the parameters $t_i,t_\oi,h_i$, $i\in[1,n]$, may be considered formally as a system of coordinates on $G^{c,c^{-1}}$ and we obtain an embedding $\CC[G^{c,c^{-1}}]\into\CC\big[t_i^{\pm1},t_\oi^{\pm1},h_i^{\pm1}:i\in[1,n]\big]$.
    Given a root $\beta=\sum\beta_i\alpha_i$, we set $t^\beta:=\prod t_i^{\beta_i}$ and $\ot^\beta:=\prod t_\oi^{\beta_i}$.

    The following factorization result is the key to constructing the cluster complex portion of the scattering diagram.
    \begin{lemma}
      For each $t\in\TT_n$ and $k\in[1,n]$, let $\beta=\bfc_{k;t}$ be the $\bfc$-vector associated to a wall in the $\bfg$-fan and consider a generic element $\lambda=\sum_{\ell\ne k}\bfg_{\ell;t}$ on this wall (this may not be a generic enough linear combination of the $\bfg$-vectors spanning this wall).
      There exists a unique factorization of $x_n(t_n)\cdots x_1(t_1)$ as $x_+ x_\beta(\pm t^\beta) x_-$ (sign to be determined),
      where $x_\pm$ lies in the subgroup of $N_+$ generated by coroot subgroups $x_\alpha(t)$ with $\pm\langle\lambda,\alpha\rangle>0$.
    \end{lemma}
    \begin{proof}
      Can we prove this using simple commutation relations?
      Maybe combined with an induction argument?

      It seems to me, quite fortunately, that we will not actually need to compute the group elements associated to imaginary walls, just avoid performing any commutations which would produce infinite factorizations since we don't need those terms.
    \end{proof}
    \begin{corollary}
      The element of $G$ associated to the wall labeled by $\beta=\bfc_{k;t}$ is precisely $x_\beta(\pm t^\beta)$.
    \end{corollary}
    
    We may consider a regular function $G^{c,c^{-1}}\to H$ alternatively as a map $\CC\big[h_i^{\pm1}:i\in[1,n]\big]\to \CC\big[t_i^{\pm1},t_\oi^{\pm1},h_i^{\pm1}:i\in[1,n]\big]$, where we abuse notation and also write $h_i$ for the coordinate function on $H$ given by $h\mapsto h^{\rho_i}$.
    For a weight $\lambda=\sum\lambda_i\rho_i$, the function $h\mapsto h^\lambda$ on $H$ is equal to $h_1^{\lambda_1}\cdots h_n^{\lambda_n}$ which we naturally abbreviate as $h^\lambda$.
    In what follows we will usually identify a Cartan-valued regular function on $G^{c,c^{-1}}$ with its value on the generic element \eqref{eq:generic element} and thus we usually denote such a function by $h:G^{c,c^{-1}}\to H$.

    The following regular functions on $G^{c,c^{-1}}$ will play a crucial role in the results below:
    \[x_{\rho_i}=h_i,\qquad z_i=t_i h_i\prod_{j<i} h_j^{a_{ji}},\qquad z_\oi=t_\oi h_i\prod_{j<i} h_j^{a_{ji}}.\]
    Observe that the functions $x_{\rho_i},z_i,z_\oi$ are algebraically independent and thus they also serve as a system of coordinates on $G^{c,c^{-1}}$, that is we have an embedding $\CC[G^{c,c^{-1}}]\into\CC\big[x_{\rho_i}^{\pm1},z_i^{\pm1},z_\oi^{\pm1}:i\in[1,n]\big]$.
    For a weight $\lambda=\sum \lambda_i\rho_i$, we write $x^\lambda=\prod x_{\rho_i}^{\lambda_i}$.
    For a root $\beta=\sum \beta_i\alpha_i\in\cQ$, we write $\hat y^\beta=\prod \hat y_i^{\beta_i}$, where 
    \[\hat y_i=z_i z_\oi \prod x_{\rho_j}^{b_{ji}}=t_i t_\oi h^{\alpha_i}.\]

  \section{Scattering Automorphisms}
    Given a root $\beta$ and a sign $\varepsilon=\pm1$, the wall crossing automorphism $\fp_{\beta,\varepsilon}$ gives the following birational automorphism of $\CC\big[t_i^{\pm1},t_\oi^{\pm1},h_i^{\pm1}:i\in[1,n]\big]$: 
    \begin{align*}
      \fp_{\beta,\varepsilon}(h^\lambda) &= h^\lambda (1+\hat y^\beta)^{\varepsilon \langle\lambda,\beta\rangle}\\
      \fp_{\beta,\varepsilon}(t^\phi) &= t^\phi (1+\hat y^\beta)^{-\varepsilon E(\beta^\vee,\phi)}\\
      \fp_{\beta,\varepsilon}(\ot^\phi) &= \ot^\phi (1+\hat y^\beta)^{-\varepsilon E(\beta^\vee,\phi)}\\
    \end{align*}

    \begin{definition}
      Given a root $\beta$ and a Cartan-valued regular function $h:G^{c,c^{-1}}\to H$, define the wall crossing function $f_\beta(h):=1+h^\beta t^\beta \ot^\beta$.
      For a root $\beta$ and a sign $\varepsilon$, define the wall crossing action on Cartan-valued regular functions $h:G^{c,c^{-1}}\to H$ as 
      \[\fp_{\beta,\varepsilon}(h)=h\cdot f_\beta(h)^{\varepsilon\beta^\vee}.\]
    \end{definition}
    \begin{lemma}
      For a root $\beta$, a choice of parameters $t_i, t_\oi$, and a regular function $h:G^{c,c^{-1}}\to H$ we have
      \[x_\obeta(\varepsilon \ot^\beta) h x_\beta(\varepsilon t^\beta) = x_\beta\big(\varepsilon h^\beta t^\beta f_\beta(h)^{-1}\big) \fp_{\beta,-}(h) x_\obeta\big(\varepsilon h^\beta \ot^\beta f_\beta(h)^{-1}\big)\]
      and
      \[x_\beta(\varepsilon t^\beta) h x_\obeta(\varepsilon \ot^\beta) = x_\obeta\big(\varepsilon h^\beta \ot^\beta f_\beta(h)^{-1}\big) \fp_{\beta,+}(h) x_\beta\big(\varepsilon h^\beta t^\beta f_\beta(h)^{-1}\big)\]
      for any choice of sign $\varepsilon$.
    \end{lemma}
    \begin{proof}
      This is a simple $SL_2$ calculation.
    \end{proof}
 
    \begin{lemma}
      Let $h:G^{c,c^{-1}}\to H$ be the regular function associated to the natural inclusion $\CC\big[h_i^{\pm1}:i\in[1,n]\big]\into\CC\big[t_i,t_\oi,h_i:i\in[1,n]\big]$.
      Let $\gamma$ be a path from the native chamber of a weight $\lambda$ to the all green chamber, say $\gamma$ crosses walls $\beta_1,\ldots,\beta_r$ with signs $\varepsilon_1,\ldots,\varepsilon_r$.
      Then $\Delta_\lambda\big(\fp_{\beta_1,\varepsilon_1}\circ\cdots\circ\fp_{\beta_r,\varepsilon_r}(h)\big)$ is equal to the cluster monomial $x_\lambda$.
    \end{lemma}
    \begin{proof}
      By induction with base case $\Delta_\lambda(h)=x^\lambda$.
      The induction step is given by the equality
      \[\fp_{\beta_{s+1},\varepsilon_{s+1}}\Big(\Delta_\lambda\big(\fp_{\beta_1,\varepsilon_1}\circ\cdots\circ\fp_{\beta_s,\varepsilon_s}(h)\big)\Big)=\Delta_\lambda\big(\fp_{\beta_1,\varepsilon_1}\circ\cdots\circ\fp_{\beta_{s+1},\varepsilon_{s+1}}(h)\big).\]
    \end{proof}



  
\end{document}
